%
% Simple template for generating drafts of papers and articles
%
\documentclass[12pt,]{article}
\usepackage[utf8]{inputenc}
\usepackage[T1]{fontenc}
\usepackage{IEEEtrantools,stackengine}
\stackMath

\usepackage{authblk}
\usepackage{fullpage}
\usepackage{amssymb,amsmath}
	\usepackage{siunitx}
\usepackage{csquotes}

\usepackage{xspace}
\newcommand\abs[1]{| #1 |}
\newcommand{\norm}[1]{\left\lVert#1\right\rVert}

% Tables
\usepackage{booktabs}
\usepackage{siunitx}
\usepackage{longtable}
\usepackage{multirow}

% Colors
\usepackage[dvipsnames]{xcolor}

\usepackage{setspace}
\doublespacing

\usepackage[unicode=true]{hyperref}
\usepackage{times}
\usepackage{float}

% Figure and Tables related;
\usepackage{graphicx}

% Plots path
\graphicspath{{./Plots/}}


\usepackage{subcaption}
\usepackage{rotating}
\usepackage{tabularx}
\usepackage{dcolumn}
\usepackage{pdflscape}
\usepackage{rotating}
\usepackage{array}

% from: http://tex.stackexchange.com/questions/2441/how-to-add-a-forced-line-break-inside-a-table-cell
\newcommand{\mcell}[2][c]{%
  \begin{tabular}[c]{@{}#1@{}}#2\end{tabular}}
\newcommand{\mcellt}[2][c]{%
  \begin{tabular}[t]{@{}#1@{}}#2\end{tabular}}
\newcommand{\lcellt}[2][l]{%
  \begin{tabular}[t]{@{}#1@{}}#2\end{tabular}}

\newcommand{\stackedcell}[2][c]{%
  \begin{tabular}[#1]{@{}c@{}}#2\end{tabular}}
	
	\usepackage{array}
\newcolumntype{L}[1]{>{\raggedright\let\newline\\\arraybackslash\hspace{0pt}}p{#1}}
\newcolumntype{C}[1]{>{\centering\let\newline\\\arraybackslash\hspace{0pt}}p{#1}}
\newcolumntype{R}[1]{>{\raggedleft\let\newline\\\arraybackslash\hspace{0pt}}p{#1}}
  
\def\click-through ratebar#1{%%
  {\color{Gray}\rule[-1.5pt]{#1cm/3}{10pt}} #1\%}


\makeatletter
\renewcommand{\fps@figure}{H}         % default {tbp}
\renewcommand{\fps@table}{H}         % default {tbp}
\makeatother 

\renewcommand{\arraystretch}{1.2}
\allowdisplaybreaks
\usepackage[margin=1.8cm]{geometry}
\pagestyle{empty}

\begin{document}
%\maketitle

\sloppy
\raggedbottom

\newpage
\begin{landscape}

\begin{table}[ht]
    \centering
    \begin{tabular}{C{.12\linewidth} | m{.5\linewidth} C{.17\linewidth}  C{.17\linewidth}}
        \toprule
        & & \emph{Supported in Experiment 1? (Minimal Groups)}  & \emph{Supported in Experiment 2? (Political Groups) }  \\ %%  \multicolumn{1}{c}{\multirow{2}{*}{\emph{Hypothesis}}} for centering hypothesis 
        \midrule
        \emph{Hypothesis 1} &  People will rate themselves and their in-group members (\emph{"Self"} \& \emph{"In-group"} Conditions) as behaving more fairly than unaffiliated others and out-group members (\emph{"Other" } \& \emph{"Out-group"} Conditions).  & No & No \\
        &&& \\
        \emph{Hypothesis 2} &  People will rate themselves (\emph{"Self"} Condition) as behaving more fairly than all others (\emph{"Other", "In-group"}, \& \emph{"Out-group"} Conditions). & No & No \\
        &&& \\
        \emph{Hypothesis 3} &  People will rate in-group members (\emph{"In-group"} Condition) as behaving more fairly than out-group members (\emph{"Out-group"} Condition). & No & \textbf{Yes} \\
        \multirow{6}{*}{\emph{Hypothesis 6}} &&& \\ 
       &   \emph{(a)} People who are highly identified with their in-group will rate in-group members  (\emph{"In-group"} Condition) as behaving more fairly. & \textbf{Yes} & No \\
       &  \emph{(b)}  People who are highly identified with their in-group will rate out-group members  (\emph{"Out-group} Condition) as behaving less fairly. & No & No \\
       &&& \\
       \midrule
        && \multicolumn{2}{c}{\emph{Supported?}} \\
      \midrule
                \emph{Hypothesis 4} &  \emph{H1}, \emph{H2}, and \emph{H3} will be confirmed in both minimal groups (\emph{Experiment 1}) and political groups (\emph{Experiment 2}) & \multicolumn{2}{c}{No} \\
      &&& \\
        \emph{Hypothesis 5} &  The effects of moral hypocrisy will be stronger for political groups (\emph{Experiment 2}) than for minimal groups (\emph{Experiment 1}) & \multicolumn{2}{c}{No} \\
        \bottomrule
    \end{tabular}
    \caption{Hypothesis table. }
    \label{hypothesis table}
\end{table}

\end{landscape}

%%   \multicolumn{n}{cols}{text}

\end{document}


